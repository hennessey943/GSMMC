\documentclass{article}
\usepackage{amsmath}
\usepackage{amssymb}
\usepackage{bm}
\usepackage{graphicx}
\usepackage{epstopdf}
\DeclareGraphicsRule{.tif}{png}{.png}{`convert #1 `basename #1 .tif`.png}
\usepackage{color}
\pagestyle{plain}
%\pagestyle{empty}
\textheight 9 true in
\textwidth 6.5 true in
\hoffset -.75 true in
\voffset -.75 true in

\mathsurround=2pt  \parskip=2pt
\def\crv{\cr\noalign{\vskip7pt}}
\def\a{\alpha } \def\b{\beta } \def\d{\delta } \def\D{\Delta } \def\e{\epsilon }
\def\g{\gamma } \def\G{\Gamma} \def\k{\kappa} \def\l{\lambda } \def\L{\Lambda }
\def\th{\theta } \def\Th{\Theta} \def\r{\rho} \def\o{\omega} \def\O{\Omega}
\def\ve{\varepsilon}
\def\p{\partial}

\def\sA{{\cal A}} \def\sB{{\cal B}} \def\sC{{\cal C}} \def\sI{{\cal I}}
\def\sR{{\cal R}} \def\sF{{\cal F}} \def\sG{{\cal G}} \def\sM{{\cal M}}
\def\sT{{\cal T}} \def\sH{{\cal H}} \def\sD{{\cal D}} \def\sW{{\cal W}}
\def\sL{{\cal L}} \def\sP{{\cal P}} \def\s{\sigma } \def\S{\Sigma}
\def\sU{{\cal U}} \def\sV{{\cal V}} \def\sY{{\cal Y}}

\def\gm{\gamma -1}
\def\summ{\sum_{j=1}^4}

\def\bb{{\bm b}} \def\yb{{\bm y}}
\def\ub{{\bm u}}  \def\xb{{\bm x}} \def\vb{{\bm v}} \def\wb{{\bm w}}
\def\omegab{{\bm \omega}} \def\rb{{\bm r}} \def\ib{{\bm i}} \def\jb{{\bm j}}
\def\lb{{\bm l}} \def\kb{{\bm k}} \def\Ab{{\bm A}} \def\fb{{\bm f}} \def\Ub{{\bm U}}
\def\Fb{{\bm F}} \def\nb{{\bm n}} \def\Db{{\bm D}} \def\eb{{\bm e}}
\def\gb{{\bm g}}  \def\Gb{{\bm G}} \def\hb{{\bm h}} \def\Yb{{\bm Y}} \def\Rb{{\bm R}}
\def\Tb{{\bm T}}

\def\As1{{\bf {\cal A}}_1}\def\DO{{\cal D}_0} \def\UO{{\cal U}_0}
\def\ie{{\it{i.e.}}}

\def\ubbar{{\bf {\bar{u}}}} \def\sbar{{\bar{\sigma }}} \def\ubar{{\bar{u}}}
\def\abar{{\bar{a}}} \def\vbar{{\bar{v}}}  \def\rbar{{\bar{\rho}}}
\def\pbar{{\bar{p}}} \def\ebar{{\bar{e}}} \def\Tbar{{\bar{T}}}
\def\bbar{{\bar{\beta}}} \def\Mbar{{\bar{M}}}  \def \sMbar{{\bar{\cal M}}}
\def\Ebar{{\bar{E}}} \def\sMbar{{\bar{\cal M}}}
\def\sPbar{{\bar{\cal P}}} \def\xbar{{\bar{x}}}

\newcommand{\pdv}[2]{\frac{\partial#1}{\partial#2}}
\newcommand{\dv}[2]{\frac{d#1}{d#2}}
\newcommand{\ord}[2]{#1^{(#2)}}
\newcommand{\vct}[1]{\vec{#1}}

 \newcommand{\bc}{\begin{center}}
 \newcommand{\ec}{\end{center}}

 \newcommand{\bq}{\begin{equation}}
 \newcommand{\eq}{\end{equation}}

 \newcommand{\beqs}{\begin{eqnarray}}
 \newcommand{\eeqs}{\end{eqnarray}}

 \newcommand{\beqa}{\begin{eqnarray*}}
 \newcommand{\eeqa}{\end{eqnarray*}}

 \newcommand{\ol}{\overline}
 \newcommand{\ul}{\underline}

 \newcommand{\dint}{{\int \!\! \int \!\!}}
 \newcommand{\tint}{{\int \!\! \int \!\! \int \!\!}}

 \newcommand{\bfig}{\begin{figure}}
 \newcommand{\efig}{\end{figure}}

 \newcommand{\cen}{\centering}
 \newcommand{\n}{\noindent}

 \newcommand{\btab}{\begin{table}}
 \newcommand{\etab}{\end{table}}

 \newcommand{\btbl}{\begin{tabular}}
 \newcommand{\etbl}{\end{tabular}}

 \newcommand{\bdes}{\begin{description}}
 \newcommand{\edes}{\end{description}}

 \newcommand{\benum}{\begin{enumerate}}
 \newcommand{\eenum}{\end{enumerate}}

 \newcommand{\bite}{\begin{itemize}}
 \newcommand{\eite}{\end{itemize}}

 \newcommand{\cle}{\clearpage}
 \newcommand{\npg}{\newpage}

 \newcommand{\bss}{\begin{singlespace}}
 \newcommand{\ess}{\end{singlespace}}

 \newcommand{\bhalf}{\begin{onehalfspace}}
 \newcommand{\ehalf}{\end{onehalfspace}}

 \newcommand{\bds}{\begin{doublespace}}
 \newcommand{\eds}{\end{doublespace}}

 \newcommand{\eps}{\mbox{$\epsilon$}}
 \newcommand{\stilde}{\mbox{$\tilde s$}}
 \newcommand{\shat}{\mbox{$\hat s$}}

 \newcommand{\blue}{\color{blue}}
 \newcommand{\red}{\color{red}}
 \newcommand{\magenta}{\color{magenta}}
 \newcommand{\green}{\color{green}}
 \newcommand{\nc}{\normalcolor}




\pagestyle{empty}
\begin{document}

\begin{center}
\large{ GSMMC \hspace{1in}  Michael Hennessey \hspace{1in}Updated May 31 2016}\end{center} 
\bigskip
Single Cell Model\\

Cell Division$\rightarrow$ Cell Expansion$\rightarrow$ Ripening\\

Focus on Cell Expansion first, then ripening, then division.\\

Loosening of cell wall due to auxin stimulated acid pump.\\

Cell swelling, fluid transport, a few key reactions, and cell wall remodeling.\\

\begin{section}{Fruit Growth post-cell division}
\begin{subsection}{Original Model}
Variables:\\
\begin{itemize}
\item $w(t)$: amount of water
\item $s(t)$ dry matter in fruit pulp
\end{itemize}
We first model the change in the amount of water in the fruit with time as the sum of the water inflow from the xylem ($U_x$) and phloem $(U_p)$ and the water outflow due to fruit transpiration ($T_f$):
\begin{equation}
\frac{dw}{dt}=U_x+U_p-T_f.
\end{equation}
The rate of change in the amount of dry matter $(ds/dt$) is the difference between the uptake from the phloem $(U_s)$ and loss through fruit respiration $(R_f$):
\begin{equation}
\frac{ds}{dt}=U_s-R_f.
\end{equation}
Fruit transpiration leading to mass loss is assumed to be proportional to the fruit surface area ($A_f)$ and to be driven by the difference in relative humidity between the air-filled space within the fruit $(H_f)$ and the ambient atmosphere $(H_a)$:
\begin{equation}
T_f=A_f\alpha\rho(H_f-H_a),
\end{equation}
where $\rho$ is the permeation coefficient of the fruit surface to water vapor and $\alpha=M_W P^*/RT,$ with $M_W=18g/mol$ being the molecular mass of water, $P^*$ the saturation vapor pressure, $R=83cm^3bar mol^{-1}K^{-1}$ the gas constant. $P^*$ is given as an exponential function
$$P^*=.009048e^{.0547(T-273.15)} bar.$$
The fruit surface area is related to the fruit total mass ($W_T)$ by the empirical equation
\begin{equation}
A_f=\gamma(W_T)^\eta.
\end{equation}
The fruit total mass is
$$W_T=s+w+W_S,$$
where $W_S$ is the fresh mass of the stone. $\gamma$ and $\eta$ depend on the fruit geometry and the density of the fruit. \\

The flow density ($J$) through a membrane is described by
\begin{equation}
J=L[P_1-P_2-\sigma(\pi_1-\pi_2)],
\end{equation}
where the subscripts 1 and 2 indicate the compartments separated by the membrane; $L$ is the hydraulic conductivity coefficient; $P_{1,2}$ are the hydrostatic pressures in the respective compartments; $\pi_{1,2}$ are the respective osmotic pressures, and $\sigma$ is the reflection coefficient (a measure of the impermeability of the membrane). The osmotic pressure is given as $\pi=RT\Sigma_jC_j^{(m)}$ in which $C_j^{(m)}=n_j(V_W^*n_W)$, where $n_j$ is the number of moles of osmotically active solute, $n_W$ is the number of moles of water, and $V_w^*$ is the partial molal volume of water. \\
The water potential $(\Psi)$ is expressed as the difference
$$\Psi=P-\pi.$$

The total flow through a membrane of area $A$ is
\begin{equation}
U=AJ.
\end{equation}

Using subscripts $x$, $p$ and $f$ for xylem, phloem and fruit variables, respectively, and combining equations 5 and 6 we get
\begin{equation}
U_x=A_xL_x[P_x-P_f-\sigma_x(\pi_x-\pi_f)]
\end{equation}
and
\begin{equation}
U_p=A_pL_p[P_p-P_f-\sigma_p(\pi_p-\pi_f)].
\end{equation}
The vascular network enters the fruit and enlarges as the fruit grows, with $A_x$ and $A_p$ increasing in parallel with fruit growth. 
$$A_x(t)=a_xA_f(t),$$
$$A_p(t)=a_pA_f(t).$$

If $\sigma_p<1$, part of the sugar can be transported from the phloem to the fruit by mass flow ($U_p$). The contribution of sugar to the mass flow is $(1-\sigma_p)C_sU_p$< where $C_s\approxeq (C_p+C_f)/2$ is the mean concentration of the solute in the membrane, with $C_p$ and $C_f$ being the sugar concentrations. The total uptake of carbohydrates is
\begin{equation}
U_s=U_a+(1-\sigma_p)C_sU_p+A_pp_s(C_p-C_f),
\end{equation}
where $p_s$ is the solute permeability coefficient. The saturating uptake rate, $U_a$, is assumed dependent on the phloem concentration according to the Michealis-Menten equation for fully non-competitive inhibition with the form
\begin{equation}
U_a=\frac{s(t)\nu_mC_p}{(K_M+C_p)(1+C_I/K_I)},
\end{equation}
where $C_I=C_I^*e^{-t/\tau}$ is the inhibitor concentration and $K_I$ is the equilibrium constant for the formation of an inhibitor-carrier complex. Here, $\nu_m$ is the maximum uptake rate per unit of dry mass and $K_M$ is the Michaelis constant. The rate $U_a$ will decline with fruit age if the inhibitor accumulates in the growing fruit.\\

The dry material loss through fruit respiration ($R_f$) comprises two components: that due to growth respiration, which is proportional to the rate of dry material intake, and that due to maintenance respiration, which is proportional to the dry mass
\begin{equation}
R_f=q_g\left(\frac{ds}{dt}\right)+q_m(T)s(t),
\end{equation}
where $q_g$ and $q_m(T)=q_m(293)Q_{10}^{(T-293)/10}$ are the coefficients for growth and maintenance respiration, respectively.
\end{subsection}
\end{section}
\end{document} 